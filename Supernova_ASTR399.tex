%[prl] it will mess up section numbers but puts our emails in the bibliography section
%[eqsecnum] Number equations by section
%
%
%\documentclass[prd,preprint,letterpaper]{revtex4}
%[nobibnotes],[nofootinbib]
\documentclass[aps,prb,twocolumn,superscriptaddress]{revtex4-1}

\usepackage{graphicx}	% this is the up-to-date package for all figures
\usepackage{amssymb}	% for math
\usepackage{verbatim}	% for the comment environment
\usepackage{color}
\usepackage{subcaption}	% for sub-captions on side-by-side figures
\usepackage{float}		% allows use of 'H' command
\usepackage{tikz}		% lets you draw: graphics, flowcharts, pie graphs, etc
\usepackage{footmisc} % to use footref, multiple footnotes referring to same thing
%\usepackage[final]{pdfpages}	% allows extern pdf to be included
\usetikzlibrary{arrows}

%%%
% For fancy section references
%%% 
% This allows you to use '\cref{}' to reference sections with the symbol
\usepackage{cleveref}
\crefname{section}{\S}{\S\S}%{§}{§§}
%% Usual (decimal) numbering
\renewcommand{\thesection}{\arabic{section}}
\renewcommand{\thesubsection}{\thesection.\arabic{subsection}}
\renewcommand{\thesubsubsection}{\thesubsection.\arabic{subsubsection}}
%% Fix references
\makeatletter
\renewcommand{\p@subsection}{}
\renewcommand{\p@subsubsection}{}
\makeatother
%%%%%
%%%


\def\deg{\ifmmode^\circ\else$^\circ$\fi}
\def\arcsec{\ifmmode^{\prime\prime}\else$^{\prime\prime}$\fi}
\def\arcmin{\ifmmode^{\prime}\else$^{\prime}$\fi}
\def\solar{\ifmmode_{\mathord\odot}\else$_{\mathord\odot}$\fi}
\def\earth{\ifmmode_{\mathord\oplus}\else$_{\mathord\oplus}$\fi}

%%%
% Change labeling of figures and tables - currently it defaults to 'FIG.'
%%%
\renewcommand{\figurename}{Figure}
\renewcommand{\thefootnote}{\roman{footnote}} %for footnotes in roman numerals
%\renewcommand{\thetable}{\Roman{table}}

\bibliographystyle{apsrev}


% these are some custom control of the page size and margins
% \topmargin= 0.2in  % these 1st two may be needed for some computers
% \textheight=8.75in
\textwidth=6.5in
%\oddsidemargin=0cm
%\evensidemargin=0cm


\begin{document}

\title{Assassinating ASASSN:\\ Supernovae Identification Using ATLAS Data}

\author{Corey Mutnik}
\email{cmutnik@hawaii.edu}
\affiliation{Department of Physics \& Astronomy, \\
University of Hawaii at Manoa,\\
2505 Correa Rd, Honolulu, HI, 96822, USA}
\altaffiliation{ASTR 399}



% \section is used to start a new one with a heading
\begin{abstract}
Using current data collection and reduction techniques, we plan to 
identify supernovae (SNe) faster and fainter than the All-Sky Automated 
Survey for Supernovae (ASASSN) team is able to. 
%With a declination limit of $-30^{\circ}$
We expect to identify all SNe, with m$\geq$17.5 and declinations above $-30\deg$, 
before ASASSN is able to.\\
%For SNe with $m>17.5$ and declination (dec) $>-30\deg$
%We expect to identify all supernovae (SN) above a declination of $-30\deg$, with $m>17.5$, before ASASSN is able to.\\
{\bf -30 or -40, JT says -40...choose one and be consistent}\\
{\bf 17.5 (0.5mag) --OR-- 18 (1mag) fainter...which is it?}\\
{\bf not emphasis on type Ia supernovae (SNIa)?}\\
{\bf Asteroid Terrestrial-impact Last Alert System (ATLAS)}
\end{abstract}


\maketitle    



\section{Introduction}
Observation of supernovae (SNe) is important in determining the distances 
of remote galaxies.  
Type Ib and Ic supernovae have been shown to vary greatly in peak brightness,{\bf cite} 
but type Ia supernova (SNIa) share the same magnitude (m) at the time of peak brightness.  
For this reason SNIa are commonly referred to as ``standard candles.''  This is 
an important feature, allowing for the calculation of distances to remote galaxies.  
Observing the violent outburst of a supernova (SN) gives us insight into our universe, 
as it was when the explosion occurred.  
%Observations of these violent explosion gives insight into the state of our universe 
%From such observations, many 
Many cosmological questions are answered by SNe observations.  Based on the current 
data, we know the universe is expanding at an accelerated rate.  Another vital 
function of SN explosions is the dispersion of heavy elements throughout the universe.  
%SNe act as mechanisms for dispersing heavy elements throughout universe.
{\bf Without SNe, the composition of our universe would be far different than we know it to be.}
{\bf The shock--wave given off by a SN explosion acts as a catalyst in the formation of stars.}~{\bf cite}


{\bf comment on the use of ATLAS data and our focus\\ATLAS is discussed in section ...}
Supernovae (SNe) identification using data collected by ATLAS.
{\bf note we plan to focus on SNIa}


%\subsection{Importance of SNe}   % THIS IS ALL COVERED IN THE INTRO SECTION
%{\bf \noindent Discuss why we care about finding SNe\\
%\indent Astronomers use SNe to ...\\
%\indent SNe are extremely important in cosmology.}




\section{Collected Data}

\subsection{ASASSN Data}

\indent The All-Sky Automated Survey for Supernovae (ASASSN) group 
collects data using eight $14~cm$ telescopes. Each night, these  
telescopes are able to cover roughly $20,000~deg^{2}$, reaching 
down to $\sim$17th magnitude.
%{\bf reword}\\
These eight telescopes are split evenly between two sites.
The first telescope array is located on Haleakala and began 
collecting data in December 2013. In July 2015, the second %telescope 
array became operational at the LCOGT Cerro Tololo station. 
This allows ASASSN to detect SNe in both 
hemispheres.~\cite{asn_data}
%{\bf Cite using footnote.--OR--~\cite{asn_data}}
~\\{\bf fix footnote:\\
\indent footnote not cite here\\
\indent footnote at bottom of this page\\
\indent fix tilde symbol in url
}



\indent Using 400~mm f/2.8G Nikon lenses allows for a large field of 
view. ProLine PL230 CCD cameras are used as detectors. Detection of 
transients is made possible using image subtraction. With images 
having 7.8\arcsec\ pixels, ASASSN relies on volunteers collecting  
confirmation images with 
larger telescopes.~\footnote{\label{foot:asn} http://www.astronomy.ohio-state.edu/ $\sim$assassin/index.shtml}



\begin{figure*}
	\centering
	\begin{subfigure}{.5\textwidth}
	  \centering
	  \includegraphics[width=1\linewidth]{figures/mjd_histo_step50.png}
		\caption{\it \small{ }}
		\label{fig:mjdhist}
	\end{subfigure}%
	\begin{subfigure}{.5\textwidth}
	  \centering
			\includegraphics[width=1\linewidth]{figures/dec_histo_step10.png}
		\caption{\it \small{ }}
		\label{fig:dechist}
	\end{subfigure}
	\caption{\it \small{SN discovered by the ASASSN project.  Panel `(a)' shows ASASSN SN discovery dates.  Panel `(b)' is ASASSN data, binned by Dec. The vertical line at $-30\deg$ indicates the lower limit on ATLAS observations.}}
	\label{fig:asnhist}
\end{figure*}


\subsection{ATLAS}
%%%%%%%%%%%%%%%%%%%%%%%%%%%%%%%%%%%%%%%%%%
%%%%%%%%%%%%%%%%%%%%%%%%%%%%%%%%%%%%%%%%%%
% Input contributions from John
%\newpage
%\input contributions/JT_portion.tex
%%%%%%%%%%%%%%%%%%%%%%%%%%%%%%%%%%%%%%%%%%
%%%%%%%%%%%%%%%%%%%%%%%%%%%%%%%%%%%%%%%%%%
ATLAS is a project funded by NASA to find dangerous asteroids.  The
motivation and science justification was described 
{\bf by~\cite{ATLAS_data}. --OR--  Tonry (2011)}.
For the duration of this ASASSN project, ATLAS used an f/2, 0.5~m
Schmidt telescope on Haleakala and a 110~Mpixel detector to collect
30~deg$^2$ with each exposure.  This telescope was installed in Jun
2015, and achieved more or less continuous operation around Sep 2015.

ATLAS will shortly install a second telescope on Mauna Loa, and a
proposal to NASA is being evaluated to build two more units for the
southern hemisphere.

The pixel size is 1.86\arcsec\ and the field of view is
5.4$\times$5.4~deg.  The PSF is currently no better than 6.5\arcsec\
which degrades the limiting sensitivity by one magnitude.  The faulty
Schmidt corrector will be replaced in March 2017.  ATLAS uses two
filters ``$o$'' (essentially $r+i$) and ``$c$'' (essentially $g+r$),
changing according to the phase of the moon.

With the basic observation consisting of a 30~sec exposure and about 
13~sec of overhead, about 900 observations are collected per night.  
%The basic observation consists of 30~sec of shutter open and about
%13~sec of overhead, so that about 900 observations are collected per night.  
The observation strategy has varied since the telescope was
installed, but currently observes one of four Dec bands between $-30$%$-40$
and $+60$ Dec, imaging each spot 5 times on a $\sim$15~min cadence.
The overlap between observations is about 0.4~deg, and they are
dithered by a random amount for a given night, so each objects is seen
5 or more times, depending on whether it falls in an overlap.  
Prior to Apr 2016, ATLAS collected 4 or more observations of each object 
per night. Increasing the number of observations per night from 4 to 5, 
while maintaining the same 4 day cadence, required an upper limit to 
be placed at a $+60$ Dec. This is shown by the dark gray region in 
the upper right hand corner of Figure~\ref{fig:dec_mjd}.

The observations are processed by the ATLAS pipeline which consists of
image flattening, star finding, star identification, high precision
fits to astrometry (typically RMS of 0.3\arcsec\ per star and more
than 10,000 stars) and photometry (typically 0.1~mag~RMS, limited by
{\bf current reference catalog)}, image differencing against a static sky
image, and detection of objects that have moved or changed.  
Difference imaging and the detection of changing objects 
are discussed in~\cref{sec:diffimg}.


\subsubsection{Difference Imaging}\label{sec:diffimg}

Difference images are generated by subtracting the wallpaper from reduced 
images. This is done to isolate objects that are changing. Subtraction 
of a properly calibrated wallpaper will produced a difference image 
containing only objects that are changing in the reduced images.  
%{\bf with no objects that are static in the reduced images.  }\\
%{\bf will remove all static objects from the difference image.  }\\

The ``wallpaper'' (static sky image) is the weighted sum of many, many
observations at each point in the sky.  ATLAS currently uses the
Alard algorithm {\bf (Alard \& Lupton, 1998)~\cite{Alard_algorithm}} for differencing, and the
differences are dominated by photon noise and by systematics from
saturated stars or flare artifacts from the Schmidt corrector.

{\bf\noindent include comparison figure?:\\ \indent red and diff imgs}

The ATLAS mission is to find moving objects, but our processing
automatically finds all stationary transients and variable stars as
well.  Once the image is differenced, the program tphot detects all
sources at 3-sigma, and cut that back to 5-sigma once each source
has been fitted and the detection significance is known.  
This is augmented by calculated RA, Dec, magnitudes, and other relevant
quantities into a ``ddt'' (difference detection table) file.

Stationary transients stand out in difference images, increasing their probability of detection.\\
{\bf other advantages of difference imaging\\}

%\begin{verbatim}
%    detection ------ real -- asteroid (moving)
%    /   \            /   \
%   /     \          /     \
%other  artifact  variable  stationary
%       /   |  \   stars    transients
%      /    |   \
%StarScar   ?    ?
%\end{verbatim}
%{\bf Make verbatism diagram into flowchart}
\begin{figure}[H]
 \resizebox{3in}{!}{%
 \tikzset{
  treenode/.style = {align=center, inner sep=1pt, font=\sffamily},
  arn_m/.style = {treenode, rectangle, rounded corners=1mm, black,
  font=\sffamily\bfseries, draw=black, minimum width=7em},
  arn_n/.style = {treenode, rectangle, black, font=\sffamily\bfseries, draw=black, minimum width=2em},
  arn_r/.style = {treenode, rectangle, black, draw=black, minimum width=3em, very thick},
  arn_x/.style = {treenode, rectangle, draw=black, minimum width=0.5em, minimum height=0.5em}
 }
 \begin{tikzpicture}[->,>=stealth',level/.style={sibling distance = 2.5cm/#1,
  level distance = 1cm}] 
 \node [arn_m] {Detections}
    child{ node [arn_r] {Real} 
            child{ node [arn_n] {Asteroid} edge from parent node[above left]
                    {$moving$}%for a named pointer
            }
            child{ node [arn_n] {Var}}
            child{ node [arn_n] {ST}}
    }
    child{ node [arn_r] {(Other)}
    }
    child{ node [arn_r] {Artifact}   
            child{ node [arn_n] {Star Scar}}
            child{ node [arn_n] {CR}}
            child{ node [arn_n] {Burn}}
            child{ node [arn_n] {xtalk}}                         
    }
 ; 
 \end{tikzpicture}%
 }%
 \caption{\small{Probability flowchart. Real - Asteroid are moving objects, ST stands for stationary transient, Var is variable star...discuss artifacts\label{fig:probflow}}}
\end{figure}



%%%%%%%%%%%%%%%%%%%%%%%%%%%%%%%%%%%%%%%%%%
%%%%%%%%%%%%%%%%%%%%%%%%%%%%%%%%%%%%%%%%%%
% Input contributions from  Ari
%\input contributions/AH_portion.tex
%%%%%%%%%%%%%%%%%%%%%%%%%%%%%%%%%%%%%%%%%%
%%%%%%%%%%%%%%%%%%%%%%%%%%%%%%%%%%%%%%%%%%
The most useful classification variable is ``starrat'' (star ratio), 
which is defined to be the ratio of flux on the original, un--subtracted 
image to flux on the difference image.  Both fluxes are measured using a 
circular aperture of 2.0~pixel radius.  The star ratio is used in 
determining if an object is real or just a residuals from subtracted stars.  
For these, the star ratio will be large (usually greater than 5) because 
the star was much brighter before it was subtracted.  For asteroids and 
supernovae, we expect the star ratio to be near 1.0.  Since they are not 
usually in their recorded locations, such objects should not show up in the wallpaper.  
%Any object that isn't in the wallpaper is left unchanged by difference imaging, causing the recorded flux to be the same both before and after image subtraction, and resulting in a starrat of 1.0.
Any object that isn't in the wallpaper is left unchanged by difference imaging.  
This means the recorded flux to be the same both before and after image subtraction, 
resulting in starrat=1.0.

{\bf possibly include starrat figure here}\\
{\bf in results section get into the range for starrat values}

Difficulties include cases where a real transient can produce a high 
starrat, and cases where a false detection from a star residual can 
produce a low starrat. The former arise from the fact that supernovae 
happen in galaxies, which do get subtracted and can raise starrat 
substantially above 1.0 if they are bright. The latter can occur when 
we have a star residual detection substantially off-center from the 
star, so the flux on the un--subtracted image is not as bright as we might 
expect, and starrat can be lowered to the 2-5 range, or in some odd 
cases can be negative. So starrat is not foolproof. Nevertheless, a 
starrat value near 1.0, especially if accompanied by other indications 
such as consistent astrometry, is a useful piece of evidence pointing 
toward a real SN, asteroid, or other interesting transient.






\section{Procedure}
%\begin{enumerate}
	%\item{} find asn SN in atlas data
	%\item{} use asn SN to restrict ATLAS classification variables
	%\item{} manually inspect results for SN classification
%\end{enumerate}
{\bf quantify: how many objects are potentially SNe before class.var. restriction, how many after\\
$0.9 < starrat < 1.2$\\}
%{\bf [check date]\\}

\indent In order to assassinate ASASSN, 
%\indent Before assassinating ASASSN became a possibility,
it was necessary to show that ATLAS had the potential to find all of ASASSN discovered SNe. 
%%To refine our SNe search, a trial set was needed. SNe ASASSN team has been 
To do this, a list of all ASASSN discovered SNe was obtained~\cite{asn_data}. 
%{\bf [--OR-- cite website using footnote]}
By Oct 11, 2016 
ASASSN has reported discovering 385 SNe. Many of these 
objects were reported before ATLAS was operation. Object cuts are discussed 
in~\cref{sec:expobs}. Once the data was properly culled, the remaining SNe 
were found in observations made by ATLAS. 
%%%% DS9 IMAGE
%One such SN is shown in Figure~\ref{fig:ds9_ASASSN-16ke}.
%%%
Finding these SNe in the ATLAS 
data allowed restrictions to be placed on classification variables, drastically 
reducing the number of potential SNe candidates. With such a restricted object 
list, visually examination is able to be used in identifying SNe.\\ 
%These identified SNe allowed for the restriction of ATLAS classification variables.

\iffalse
 %{\bf use DS9 image? if so, then:\\
 %\indent remake it\\
 %\indent identify exact dates shown by tiles}

 \begin{figure}[H]%[h!]
    \begin{center}
  \centerline{\includegraphics[width=3.35in]{figures/ds9_asn16ke.png}}
  \caption{\it \small{Each tile shows a different ATLAS observation of the SN ``ASASSN-16ke''. The SN is enclosed by a green circle, to show its exact location. \label{fig:ds9_ASASSN-16ke}}}
    \end{center}
 \end{figure}
 {\bf comment on lower panel showing ATLAS obs before SN appeared in fig~\ref{fig:ds9_ASASSN-16ke}?}
\fi


%%%%%%%%%%%%%%%%%%%%%%%%%%%%%%%%%%%%%%%%%%%%%%%%%%%%%%%%%%%%%%%%%%%%%%%%%%%%%%%%%%%%%%%%%%%%%%%%%%%%
%%%%%%%%%%%%%%%%%%%%%%%%%%%%%%%%%%%%%%%%%%%%%%%%%%%%%%%%%%%%%%%%%%%%%%%%%%%%%%%%%%%%%%%%%%%%%%%%%%%%
%%%
% input 2 sections: ``Expected Observations'' and ``Failed Matches''
%%%
\input sections/section_expected_obs.tex
%%%%%%%%%%%%%%%%%%%%%%%%%%%%%%%%%%%%%%%%%%%%%%%%%%%%%%%%%%%%%%%%%%%%%%%%%%%%%%%%%%%%%%%%%%%%%%%%%%%%
%%%%%%%%%%%%%%%%%%%%%%%%%%%%%%%%%%%%%%%%%%%%%%%%%%%%%%%%%%%%%%%%%%%%%%%%%%%%%%%%%%%%%%%%%%%%%%%%%%%%



\section{Results and Discussion}

ATLAS had the potential to observe 161 of the 385 SNe ASASSN discovered 
by Oct 2016.  Before ATLAS was truly operational, another 65 SNe had 
peaked, leaving 96 objects.  
{\bf All 96 SNe were found using ATLAS. }\\
{\bf verify all 96 found with $\pm$day window?}\\
{\bf merge with para below}

\indent It has been shown that all ASASSN discovered SNe overlapping ATLAS 
observations in sky and time are detectable in ATLAS data. Restricting 
classification variables drastically reduces the false alarm rate, requiring 
visually inspection of fewer objects. This shows the ability for ATLAS to 
identify all SNe faster and fainter than ASASSN is able to.
%
\indent The detection of all SNIa between $-30$ and 
$+60$ Dec with m$\geq$17.5 is possible using ATLAS data.  
%As shown by Figure~\ref{fig:dec_mjd}, ATLAS was able to identify all of the SNe discovered by ASASSN
All SNe discovered by ASASSN, after ATLAS became truly operational, were 
able to be identified with ATLAS data alone.  This is shown in 
Figure~\ref{fig:dec_mjd}, with missed SNe falling on the edges of ATLAS 
observational limits.
%
\indent ASASSN discovered SNe were used in refining ATLAS classification 
variables, such as starrat.  Variables like starrat reduce the 
false alarm rate by eliminating artifacts produced during image 
subtraction.  
%{\bf Detection of all SNIa is possible using ATLAS.\\}
%{\bf As seen in Figure~\ref{fig:dec_mjd}, SNe that were not 
%observed ATLAS fall on the edges of observation limits.\\}
%\section{Results}
%{\bf Possibly merge these two sections into single: ``Results and Discussion''}\\
{\bf Summary of data matched between ASASSN and ATLAS.\\
reference sections that explain particular cases that matching failed}
%\subsection{Independent SN Identification}
%\subsection{SN Identification}
%\section{Discussion}
\begin{enumerate}
  \item{} How ASASSN SN help identify those in ATLAS data.
	\item{} what restrictions we intend to place on classification variables like starrat
	%\item{} possibly describe what starrat is, how it will help id SN
	\item{} such restrictions cut the number of objects down from xx to $\sim$1000/2000
  \item{} \indent quantify above, using cm groupings number
  \item{} ref: \cref{sec:expobs} and \cref{sec:failmatch}
\end{enumerate}








{\bf \noindent BELOW IS FROM PREVIOUS ATLAS SECTION (written by me)
\indent possibly place it here or in procedure section}\\
Requiring an object shows up on more than one image recorded in the same night 
drastically reduces our false alarm rate.\\
An object is determined to be real if it appears in all good 
observations overlapping that region of the sky. 
If an object appears in one of the images, it is expected to appear 
in the other overlapping images collected that night.
% images collected that night that overlap the same region on the sky. 
This {\bf expectation / requirement} drastically reduces the false alarm rate.\\
\indent With a relatively short average lifetime, detection of SNIa requires 
frequent coverage of the entire sky. Using current observation patterns,
ATLAS surveys the entire sky, between $-30$ and $+60$ Dec, in just 4 nights. 
Five observations of the entire sky are recorded by ATLAS every four nights, 
making detection of all SNIa possible.




\section*{Acknowledgments}
I would like to thank \input acknowledgement.tex  % input acknowledgement



%%%%%%%%%%%%%%%%%%%%%%%%%%%%%%%%%%%%%%%%%%%%%%%%%%%%%%%%%%%%%%%%%%%%%%%%%%%%%%%%%%%%%%%%%%%%%%%%%%%%
%%%%%%%%%%%%%%%%%%%%%%%%%%%%%%%%%%%%%%%%%%%%%%%%%%%%%%%%%%%%%%%%%%%%%%%%%%%%%%%%%%%%%%%%%%%%%%%%%%%%
%%%
% Input general notes and things that need to be fixed / addressed
% REMOVE ONCE FINISHED
%%%
\clearpage
\begin{widetext} % make something one column...usually used for long equations...works, but resize looks better
 \section{General Notes \& Things to Address}
 \input general_notes.tex\\
 {\bf make sure the acronym ``ATLAS'' is defined\\}
 {\bf check plurality throughout paper: SN or SNe\\}
 {\bf make sure acronyms are only defined once\\}
 {\bf Verify Tonry et al. (2011) journal is cited properly: PASP\\}
 {\bf Check JT's papers for proper capitalization:\\
 \indent Type Ia Supernovae\\
 \indent Type Ia supernovae\\
 \indent type Ia Supernovae\\
 \indent type Ia Supernovae\\}
 {\bf ``ASASSN'' --OR-- ``ASAS--SN''\\}
 {\bf make sure all paragraphs are broken-up and indented as desired\\}
 {\bf 1.86\arcsec\ looks better than 1.86\arcsec does...change all to have backslash at the end}\\
 {\bf declination (Dec) NOT: dec}\\
 %{\bf m$>$17.5 --OR-- $m>17.5$\\
 %\indent USE: m$>$17.5}\\
 {\bf change ``diff'' to difference images, --OR-- define it}\\ 
 {\bf check email if JT \& AH want to be authors or acknowledgments}\\
 {\bf check JT paper to see if authors/acknowledgments \\
 \indent how they were thanked\\
 \indent include the institution where the person works}\\
 %{\bf make sure all figures are referenced and done so properly}
 {\bf Necessary to define MJD?}\\
 {\bf check floats [H]}\\
 {\bf All months abbreviated...consistency}
\end{widetext}
%%%%%%%%%%%%%%%%%%%%%%%%%%%%%%%%%%%%%%%%%%%%%%%%%%%%%%%%%%%%%%%%%%%%%%%%%%%%%%%%%%%%%%%%%%%%%%%%%%%%
%%%%%%%%%%%%%%%%%%%%%%%%%%%%%%%%%%%%%%%%%%%%%%%%%%%%%%%%%%%%%%%%%%%%%%%%%%%%%%%%%%%%%%%%%%%%%%%%%%%%

%%%%%%%%%%%%%%%%%%%%%%%%%%%%%%%%%%%%%%%%%%%%%%%%%%%%%%%%%%%%%%%%%%%%%%%%%%%%%%%%%%%%%%%%%%%%%%%%%%%%
%%%%%%%%%%%%%%%%%%%%%%%%%%%%%%%%%%%%%%%%%%%%%%%%%%%%%%%%%%%%%%%%%%%%%%%%%%%%%%%%%%%%%%%%%%%%%%%%%%%%
%%%
% Input variations in plot
% REMOVE ONCE FINISHED
%%%
%\newpage
%\input figures/variations/variations.tex
%\newpage
%%%%%%%%%%%%%%%%%%%%%%%%%%%%%%%%%%%%%%%%%%%%%%%%%%%%%%%%%%%%%%%%%%%%%%%%%%%%%%%%%%%%%%%%%%%%%%%%%%%%
%%%%%%%%%%%%%%%%%%%%%%%%%%%%%%%%%%%%%%%%%%%%%%%%%%%%%%%%%%%%%%%%%%%%%%%%%%%%%%%%%%%%%%%%%%%%%%%%%%%%


\iffalse
 %%%%%%%%%%%%%%%%%%%%%%%%%%%%%%%%%%%%%%%%%%%%%%%%%%%%%%%%%%%%%%%%%%%%%%%%%%%%%%%%%%%%%%%%%%%%%%%%%%%
 %%%%%%%%%%%%%%%%%%%%%%%%%%%%%%%%%%%%%%%%%%%%%%%%%%%%%%%%%%%%%%%%%%%%%%%%%%%%%%%%%%%%%%%%%%%%%%%%%%%
 %\clearpage
 \newpage
 \section{REMOVE THESE NOTES WHEN FINISHED}
 %\input JT_meeting_161115_notes.tex 
 \includepdf[pages=-]{JT_meeting_161115_notes.pdf}
 %%%%%%%%%%%%%%%%%%%%%%%%%%%%%%%%%%%%%%%%%%%%%%%%%%%%%%%%%%%%%%%%%%%%%%%%%%%%%%%%%%%%%%%%%%%%%%%%%%%
 %%%%%%%%%%%%%%%%%%%%%%%%%%%%%%%%%%%%%%%%%%%%%%%%%%%%%%%%%%%%%%%%%%%%%%%%%%%%%%%%%%%%%%%%%%%%%%%%%%%
\fi



\setlength{\parindent}{0cm}

\bibliography{biblio}

%\begin{thebibliography}{99}  % the trailing 99 controls some obscure format--just use
%\bibitem{Sch_eq} Weisstein, Eric W. ``Schr\"{o}dinger Equation."
%{\em Schr\"{o}dinger Equation. } Mathematica, 1996. Web. 2 May 2015.
%\end{thebibliography}

\end{document}
