%The most useful identification variable is the starratio (starrat), 
%which is defined to be the ratio of flux on the original, unsubtracted 
%image to flux on the difference image.
starrat (short for star ratio) is the ratio of flux on the original, 
unsubtracted image to flux on the difference image. Both fluxes are 
measured in a circular aperture of 2.0~pixel radius. The star ratio 
is useful because it helps us weed out detections that are not real 
objects but just residuals from subtracted stars. For these, the star 
ratio will be large (usually greater than 5) because the star was much 
brighter before it was subtracted. For asteroids and supernovae, we 
expect the star ratio to be near 1.0: the object isn't usually there, 
and so it is not in the wallpaper and will not be subtracted at all: 
its flux will be the same both before and after image subtraction, so starrat=1.0.

Difficulties include cases where a real transient can produce a high 
starrat, and cases where a false dectection from a star residual can 
produce a low starrat. The former arise from the fact that supernovae 
happen in galaxies, which do get subtracted and can raise starrat 
substantially above 1.0 if they are bright. The latter can occur when 
we have a star residual detection substantially off-center from the 
star, so the flux on the unsubtracted image is not as bright as we might 
expect, and starrat can be lowered to the 2-5 range, or in some odd 
cases can be negative. So starrat is not foolproof. Nevertheless, a 
starrat value near 1.0, especially if accompanied by other indications 
such as consistent astrometry, is a useful piece of evidence pointing 
toward a real supernova, asteroid, or other interesting transient.