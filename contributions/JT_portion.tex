ATLAS is a project funded by NASA to find dangerous asteroids.  The
motivation and science justification was described by {\bf Tonry (2011)}.
For the duration of this ASASSN project, ATLAS used an f/2, 0.5~m
Schmidt telescope on Haleakala and a 110~Mpixel detector to collect
30~deg$^2$ with each exposure.  This telescope was installed in Jun
2015, and achieved more or less continuous operation around Sep 2015.

ATLAS will shortly install a second telescope on Mauna Loa, and a
proposal to NASA is being evaluated to build two more units for the
southern hemisphere.

The pixel size is 1.86\arcsec\ and the field of view is
5.4$\times$5.4~deg.  The PSF is currently no better than 6.5\arcsec\
which degrades the limiting sensitivity by one magnitude.  The faulty
Schmidt corrector will be replaced in March 2017.  ATLAS uses two
filters ``$o$'' (essentially $r+i$) and ``$c$'' (essentially $g+r$),
changing according to the phase of the moon.

The basic observation consists of 30~sec of shutter open and about
13~sec of overhead, so that about 900 observations are collected per
night.  The observation strategy has varied since the telescope was
installed, but currently observes one of four Dec bands between $-40$
and $+60$ Dec, imaging each spot 5 times on a $\sim$15~min cadence.
The overlap between observations is about 0.4~deg, and they are
dithered by a random amount for a given night, so each objects is seen
5 or more times, depending on whether it falls in an overlap.

The observations are processed by the ATLAS pipeline which consists of
image flattening, star finding, star identification, high precision
fits to astrometry (typically RMS of 0.3\arcsec\ per star and more
than 10,000 stars) and photometry (typically 0.1 mag RMS, limited by
current reference catalog), image differencing against a static sky
image, and detection of objects that have moved or changed.

The ``wallpaper'' (static sky image) is the weighted sum of many, many
observations at each point in the sky.  ATLAS currently uses the
Alard algorithm {\bf (Alard \& Lupton, 199x)} for differencing, and the
differences are dominated by photon noise and by systematics from
saturated stars or flare artifacts from the Schmidt corrector.

The ATLAS mission is to find moving objects, but our processing
automatically finds all stationary transients and variable stars as
well.  Once the image is difference, the program tphot detects all
sources at 3-sigma, and cut that back to 5-sigma once each source
has been fitted and the detection significance is known.  This
is augmented by calculated RA, Dec, magnitudes, and other relevant
quantities into a ``ddt'' (difference detection table) file.