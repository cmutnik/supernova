%[prl] it will mess up section numbers but puts our emails in the bibliography section
%[eqsecnum] Number equations by section
\documentclass[aps,prb,twocolumn,superscriptaddress]{revtex4-1}
\usepackage{graphicx}  % this is the up-to-date package for all figures
\usepackage{amssymb}   % for math
\usepackage{verbatim}  % for the comment environment
\usepackage{color}
\usepackage{subcaption} % for subcaptions on side-by-side figures
\usepackage{float}	% allows use of 'H' command

% This allows you to use '\cref{}' to reference sections with the symbol
\usepackage{cleveref}
\crefname{section}{\S}{\S\S}%{§}{§§}


%\usepackage{cite}	% to use .bib file
%\usepackage{hyperref}	% needed to add hyperlinks
%\hypersetup{
%  colorlinks=true,
%  linkcolor=blue,
%  filecolor=magenta,
%  urlcolor=cyan,
%}
\bibliographystyle{apsrev}

% For Numbered Sections---------------------------------------------------------------
% Usual (decimal) numbering
\renewcommand{\thesection}{\arabic{section}}
\renewcommand{\thesubsection}{\thesection.\arabic{subsection}}
\renewcommand{\thesubsubsection}{\thesubsection.\arabic{subsubsection}}
%% Fix references
%\makeatletter
%\renewcommand{\p@subsection}{}
%\renewcommand{\p@subsubsection}{}
%\makeatother
% ------------------------------------------------------------------------------------


% these are some custom control of the page size and margins
% \topmargin= 0.2in  % these 1st two may be needed for some computers
% \textheight=8.75in
\textwidth=6.5in
%\oddsidemargin=0cm
%\evensidemargin=0cm

% this is where the actual document itself (rather than control statements) begins:

\begin{document}



\title{Evaluating the Pan-STARRS Variability Parameter}


%\input author_list.tex

%\author{Daichi Hiramatsu\thanks{dhiramat@hawaii.edu}}
%\author{Corey Muntik\thanks{cmutnik@hawaii.edu}}

\author{Daichi Hiramatsu}
\email{dhiramat@hawaii.edu}
\author{Corey Mutnik}
\email{cmutnik@hawaii.edu}
\affiliation{Department of Physics \& Astronomy, \\
University of Hawaii at Manoa}
%\affiliation{Department of Physics \& Astronomy, \\University of Hawaii at Manoa,\\2505 Correa Rd, Honolulu, HI, 96822, USA}
%\altaffiliation{Observational Astronomy 301}



\begin{abstract}
\textbf{By Thursday (4/18) we need:} well thought out section titles and plots that show all the points we wanna make\\

remake prob(f) plot with all 300,000 stars (not only 80,000)\\

FLS analysis on ATLAS Pathfinder Telescope data, verified PS variability criteria
\end{abstract}

\maketitle    




\section{Introduction}
Hello DH~\cite{RRLyrae}
\begin{itemize}
	\item{} why we care
	\item{} what made us care about this project
	\item{} NO structure / distance stuff (maybe put it in looking forward section at end)
	\item{} talk about PS catalog
	\item{} variability surveys (discuss other attempts to measure variables across the sky)
	\item{} why are variables interesting
	\item{} why do we want to find variables and care about where they are located
	\item{} Summary: we ran FLS, analyzed stars, why did we do it all
	\item{} Mention what will be discussed: ``in section 2 we describe the observations we used...''
\end{itemize}



\section{ATLAS PathFinder 1 Observations}
\begin{itemize}
	\item{} we used data from ATLAS
	\item{} supplemented with ATLAS data [REF TONRY] (possibly make this s subsection)
%	\item{} exposure time
%	\item{} observation dates
	\item{} what was the weather like during observations
	\item{} PSF FWHM variations (only include if we discuss crowding)
	\item{} `we recieved the reduced image data from the ATLAS pipeline; which gave us RA, Dec, mag, etc...'
	\item{}\url{http://fallingstar.com/how_atlas_works.php}
\end{itemize}


~[VERIFY CORRECT CITATIONS:\\% $http://rsta.royalsocietypublishing.org/content/371/1992/20120269$]\\
%In determining the variability of stars in our FOV, data from the $gri Project$~\cite{gri} was used.\\
Initially, determination of variability was going to be achieved using data collected by the $gri Project$~\cite{gri}.  The $gri Project$ is [EXPLAIN]...\\
In order to reduce aliasing, extra observations needed to be made.  Observation procedures are discussed in~\cref{sec:data}.
~[PATHFINDER USED FOR GRI DATA...the reduction process is discussed at length Tonry in...cite]\\
We received the reduced image data from the ATLAS pipeline~\cite{gri}~\cite{tonrypipe}
%\cite{PSimgpipe,*tonrypipe}


\subsection{Data Collection}\label{sec:data}
%\begin{itemize}
%	\item{} how we got mags out of data...
%	\item{} $l^{II}=202^{\circ}$
%	\item{} $b^{II}=\pm5$
%\end{itemize}
% from old paper
\iffalse
	\begin{itemize}
	 \item{} split observations into 1 $deg^2$ chunks
	 \item{} isolated groups s.t. each one is a star with 12 or more obs
	 \item{} $--$ more than 12 detections to be a star
	 \item{} $--$ any sq deg that has more than one star
	 \item{} this reduced 1300 $deg^2$ observation data down to ~300
	 \item{} before variability params: 1531417 stars in field
	 \item{} for variability parameters
	 \item{} $--$ log(average(upper quartile)) - log(average(lower quartile))
	 \item{} $--$ expect variation to go at .2* mag (from sqrt noise)...so subtract .2mag to get the logritmic statistic
	 \item{} sorted biggest (most variable?) to smallest (least variable?)
%	 \item{} ran FLS on 80,000 most variable stars, rather than full star groupings (over 1million)
	\end{itemize}
\fi

~[NEED TO CITE TONRY FOR OUR OBSERVATIONS]\\
~[MENTION THAT OUR OBS NECESSARY TO REDUCE ALIASING]\\

Using the Pathfinder telescope, observations were made at two galactic latitudes ($b^{II}=\pm5^{\circ}$) and spanned a range of galactic longitudes ($202^{\circ} < l^{II} < 232^{\circ}$).  Observations discussed here indicated the center of each FOV.  
Exposures were collected for 20~s and separated by $3^{\circ}$ longitudinally.  For implementation by the Pathfinder telescope, a conversion to RA and Dec was made; giving a range of $93 < RA < 119^{\circ}$ and $-20^{\circ} < Dec < 13^{\circ}$, as shown by Figure~\ref{fig:simoverlap}.  To account for the 0.05~$^{\circ}$ gap between the detectors, a 0.1~$^{\circ}$ offset in RA was implemented on every other night.  
%Ten observations per night were collected for 20 nights, spanning 3/8/16-3/27/16.
%Observations were collected between 3/8/16-3/27/16.  During this 20 day period, we collected 10 observations per night.  
Spanning 20 nights, 10 observations a night were collected on 3/8/16-3/27/16.  Luckily, all of these nights had weather perfectly attuned for observations.  Half a night of observations were lost on 3/19/26, due to a crash of the server controlling the telescope.


%0.1 deg offset in RA, accounts for 0.05 deg gap between the two detectors.
%~\\~[dRA = dangle / cos(Dec)]
Observations were traced out by moving the FOV by $3^{\circ}$ longitudinally, starting at $b^{II}=-5^{\circ}$, $l^{II}=202^{\circ}$ and ending at $b^{II}=-5^{\circ}$, $l^{II}=232^{\circ}$.  Once observations at $b^{II}=-5^{\circ}$ were complete, the FOV was shifted to $b^{II}=+5^{\circ}$, $l^{II}=232^{\circ}$.
%Observations were traced out by taking the first exposure at $b^{II}=-5^{\circ}$ and $l^{II}=202^{\circ}$, moving the FOV longitudinally by $3^{\circ}$
%Starting at $b^{II}=-5^{\circ}$ and $l^{II}=202^{\circ}$, exposures were collected for 20~s
%Observations were made by starting at $b^{II}=-5^{\circ}$ and $l^{II}=202^{\circ}$, moving the ...Each exposure was collected for 20
%Data was collected for \\
%Starting at $b^{II}=-5^{\circ}$ and $l^{II}=202^{\circ}$ \\
%20~s exposures were collected\\


%\begin{figure}[H]
%	\centering
%	%\caption{prob(f)}
%	\textbf{Sorting Pattern}\par\medskip
%		\includegraphics[width=3.35in]{figures/fromJT/sortedFOV.png}
%	\caption{\it \small{Stars were grouped in the pattern shown (down the collected observations)}}
%	\label{fig:sortpat}
%\end{figure}



\subsection{Object Cuts}\label{sec:cuts}

Various data points, reduced by the ATLAS pipeline~\cite{gri}, were returned with magnitude errors of zero.  Such 
values were not used determining variability of the source.  In order for an identified star to be considered for 
variability testing, we required a minimum of 12 ``good'' observations.  A ``good'' detection is meets the minimum 
PSF, does not fall on the edge of each $1~deg^{2}$ FOV, and was observed with clear skies.  A minimum of 12 
observations was deemed necessary, in order to eliminate aliasing, as discussed in~\cref{sec:PA}.  After observation 
cuts were applied, 1.5 Billion stars remained in our FOV.  


\begin{figure}[H]
 \centering
 	\includegraphics[width=3.35in]{figures/NEW/scatter_gfilt_57452_57455.png}
 \caption{\it \small{Lower quartile variance as a function of magnitude.}}
 \label{fig:probrrHPS}
\end{figure}
  In order to exploit the lower quartile variance, shown in Figure~\ref{fig:quart}, a factor of $0.2m$ was added to account for Poisson error.  
The higher a star deviates from the average value, the more likely it is variable.  This flagged 320,000, of the observed 1.5 Billion, for 
further variability verification.


As a result of running FLS, the most variable candidates fell below $logPr(rnd)\leq-45$, shown in Figure~\ref{fig:logPr}.  An in depth discussion how $logPr(rnd)$ works can be found in \textit{Numerical Recipes}~\cite{logPr, Numerical}.  Using $logPr(rnd)$ made it impossible 
to miss any variable star with $0.05~\leq~Period~\leq~1.2$~\cite{AAVSO}.  It is statistically improbable that any observed variable star wasn't flagged.  If any variable stars were not identified, they must have lower period variations and are therefore not classified as RR Lyrae.




\iffalse
of all 1.5 billion stars in our FOV we scanned the most variable $(logPr(rnd)<=45)$\\
  - located: ${{{cd /home/rr_lyrae/logprob_below_neg45_take2}}}$\\
  - 1558 stars\\
new grouping list that DH wanted (after removing aliased regions)\\
  - located: ${{{cd /home/rr_lyrae/alias_rm_grps}}}$\\
  - 776 stars\\
\fi


\section{Constructing Stellar Light Curves}

% check out resource:
% https://www.aavso.org/lcg

\begin{itemize}
	\item{} how we selected stars (12+ obs, 1x1 deg$^2$, etc)
\end{itemize}

The selection process began


% merged these figures into side by side (below)
%\begin{figure}[H]
% \centering
% 	\includegraphics[width=3.35in]{figures/LC/star7_97_98_0_1.png}
% \caption{\it \small{Light Curve of variable star, using all collected and ATLAS data.}}
% \label{fig:LC7}
%\end{figure}
%\begin{figure}[H]
% \centering
% 	\includegraphics[width=3.35in]{figures/LC/star7_97_98_0_1_restricted.png}
% \caption{\it \small{Light Curve of variable star, using only collected data (without incorporating extra ATLAS data).}}
% \label{fig:restrictedLC7}
%\end{figure}



\begin{figure*}
	\centering
	\begin{subfigure}{.5\textwidth}
	  \centering
	  \includegraphics[width=1\linewidth]{figures/LC/star7_97_98_0_1.png}
		\caption{\it \small{ }}
		\label{fig:LC7}
	\end{subfigure}%
	\begin{subfigure}{.5\textwidth}
	  \centering
			\includegraphics[width=1\linewidth]{figures/LC/star7_97_98_0_1_restricted.png}
		\caption{\it \small{ }}
		\label{fig:restrictedLC7}
	\end{subfigure}
	\caption{\it \small{Light Curve of a variable star.  Panel `(a)' shows a light curve constructed using all collected and ATLAS data.  Panel `(b)' is a restricted selection of `(a)', not showing any observations made by ATLAS.}}
	\label{fig:LC}
\end{figure*}




%\section{Evaluating the Pan-STARRS Variability Parameter}
\section{Evaluating the HPS Variability Parameter}
\iffalse
	\begin{itemize}
	%	\item{} compare with HPS~\cite{PSdata}
	%	\item{} density/distribution of variables in sky
	%	\item{} (what FLS gave us for our catalog)
		\item{} put a table with ~5 stars, to show off part of catalog
	\end{itemize}


	\begin{table}[H]
		\begin{center}
			\begin{tabular}{|c|c|c|c|c|}\hline
				HPS $\rho_{RR}$	&	nonvar15	&	var15	&	nonvar20	&	var20 \\ \hline
				0.0-0.05 	&	89 			&	41 		&	104 		&	26 \\ \hline
				0.05-0.1 	&	13 			&	11 		&	18 			&	6 \\ \hline
				0.1-0.2 	&	18 			&	16 		&	25 			&	9 \\ \hline
				0.2-0.3 	&	20 			&	12 		&	23 			&	9 \\ \hline
				0.3-0.4 	&	14 			&	21 		&	24 			&	11 \\ \hline
				0.4-0.5 	&	13 			&	19 		&	21 			&	11 \\ \hline
				0.5-0.6 	&	18 			&	21 		&	25 			&	14 \\ \hline
				0.6-0.7 	&	17 			&	25 		&	23 			&	19 \\ \hline
				0.7-0.8 	&	11 			&	27 		&	19 			&	19 \\ \hline
				0.8-0.9 	&	3 			&	21 		&	8 			&	16 \\ \hline
				0.9-1.0 	&	6 			&	7 		&	6 			&	7 \\ \hline
			\end{tabular}
		\end{center}
	  \caption{ \small{ProbRR for each bin, for limit 0.15 and limit 0.2\label{tab:probbRRbin}}}
	\end{table}	

	\begin{table}[H]
		\begin{center}
			\begin{tabular}{|c|c|c|}\hline
			%HPS stat & nonvar20 & var20 \\ \hline
			HPS stat & Not Variable & var20 \\ \hline
			0.0-0.05 & 104 & 26 \\ \hline
			0.05-0.1 & 18 & 6 \\ \hline
			0.1-0.2 & 25 & 9 \\ \hline
			0.2-0.3 & 23 & 9 \\ \hline
			0.3-0.4 & 24 & 11 \\ \hline
			0.4-0.5 & 21 & 11 \\ \hline
			0.5-0.6 & 25 & 14 \\ \hline
			0.6-0.7 & 23 & 19 \\ \hline
			0.7-0.8 & 19 & 19 \\ \hline
			0.8-0.9 & 8 & 16 \\ \hline
			0.9-1.0 & 6 & 7 \\ \hline
			\end{tabular}
		\end{center}
	\caption{ \small{RR Table 0.2 \label{tab:HPSlim20}}}
	\end{table}

	HPSinFOV = [5029, 124, 154, 116, 82, 85, 90, 89, 64, 46, 21]
	% lim 0.20
	RRlim15vals= [46, 12, 19, 15, 22, 20, 22, 28, 28, 23, 9, 0]
	notvariable15 = [92, 13, 19, 21, 14, 14, 19, 19, 11, 5, 6, 0]
	% lim 0.20
	notvariable20=[110, 18, 27, 27, 25, 22, 27, 26, 20, 10, 7, 0]
	RRlim20vals =  [28, 7, 11, 9, 11, 12, 14, 21, 19, 18, 8, 0]
\fi

A paper, \textit{Finding, Characterizing and Classifying Variable Sources in Multi-Epoch Sky Surveys: QSOs and RR Lyrae in PS1 $3\pi$ Data}~\cite{PSdata} (HPS), quantifies the likelihood that a star is an RR Lyrae.  Using their variability statistic, $\rho_{RRLyrae}$, density and distribution of RR Lyrae and other variable candidates were determined.  Table~\ref{tab:HPSlim15} evaluates the validity of HPS's variability criteria.
\iffalse
	% HPS.restrictFOV.m11.m16.chyea has 5900 stars in it --> 5900 HPS stars in our FOV
	% THIS TABLE IS FOR LIMIT 0.15
	\begin{table*}
	%\begin{table}[H]
	\begin{center}
		\begin{tabular}{|c|c|c|c|c|c|c|c|c|}\hline
			%HPS stat & nonvar15 & var15 \\ \hline
	HPS $\rho_{RRLyrae}$ & $HPS_{total}$ & $HPS_{matched}$ & RR & var (notRR) & $Not-Variable_{matched}$ &$notvar_{notmatched}$(outside mask)& Prob(var)&Prob(RR) \\ \hline
	0.0-0.05 & 5029 & 138 & 9 & 37 & 92 	& 4891 & $46/138=0.33$ & $9/138=0.07$	\\ \hline
	0.05-0.1 & 124 & 25 & 1 & 11 & 13 		& 99 & 12/25 & 1/25 \\ \hline
	0.1-0.2 & 154 & 38 & 1 & 18 & 19 		& 116 & 19/38 & 1/38 \\ \hline
	0.2-0.3 & 116 & 36 & 1 & 14 & 21 		& 80 & 15/36 & 1/36 \\ \hline
	0.3-0.4 & 82 & 36 & 1 & 21 & 14 		& 46 & 22/36 & 1/36	\\ \hline
	0.4-0.5 & 85 & 34 & 1 & 19 & 14 		& 51 & 20/34 & 1/34	\\ \hline
	0.5-0.6 & 90 & 41 & 5 & 17 & 19 		& 49 & 19/41 & 5/41	\\ \hline
	0.6-0.7 & 89 & 47 & 6 & 22 & 19 		& 42 & 28/47 & 6/47	\\ \hline
	0.7-0.8 & 64 & 39 & 5 & 23 & 11 		& 25 & 28/39 & 5/39	\\ \hline
	0.8-0.9 & 46 & 28 & 4 & 19 & 5 			& 18 & 23/28 & 4/28	\\ \hline
	0.9-1.0 & 21 & 15 & 1 & 8 & 6 			& 6 & 9/15 & 1/15	\\ \hline
	\hline
	0.0-1.0 & 5900 & 477 & 35 & 209 & 233 	& 5423 & & \\ \hline
		\end{tabular}
	\end{center}
	\caption{ \small{A comparison of observations to HPS RR Lyrae candidates. \label{tab:HPSlim15}}}
	\end{table*}
\fi
% HPS.restrictFOV.m11.m16.chyea has 5900 stars in it --> 5900 HPS stars in our FOV
% THIS TABLE IS FOR LIMIT 0.15
\begin{table*}
%\begin{table}[H]
	\begin{center}
		\begin{tabular}{|c|c|c|c|c|c|c|c|}\hline
			%HPS stat & nonvar15 & var15 \\ \hline
HPS $\rho_{RRLyrae}$ & $HPS_{total}$ & $HPS_{matched}$ & RR Lyrae & Not RR Lyrae & Prob(RR) & Prob(notRR) \\ \hline
0.0-0.05 & 5029 & 138 & 46 		& 92 & 0.33 & -- \\ \hline
0.05-0.1 & 124 & 25 & 12 		& 13 & 0.48 & -- \\ \hline
0.1-0.2 & 154 & 38 & 19 		& 19 & 0.50 & -- \\ \hline
0.2-0.3 & 116 & 36 & 15 		& 21 & 0.42 & -- \\ \hline
0.3-0.4 & 82 & 36 & 22 			& 14 & 0.61 & -- \\ \hline
0.4-0.5 & 85 & 34 & 20			& 14 & 0.59 & -- \\ \hline
0.5-0.6 & 90 & 41 & 22			& 19 & 0.54 & -- \\ \hline
0.6-0.7 & 89 & 47 & 28			& 19 & 0.60 & -- \\ \hline
0.7-0.8 & 64 & 39 & 28 			& 11 & 0.72 & -- \\ \hline
0.8-0.9 & 46 & 28 & 23 			& 5 & 0.82 & -- \\ \hline
0.9-1.0 & 21 & 15 & 9			& 6 & 0.60 & -- \\ \hline
\hline
0.0-1.0 & 5900 & 477 & 244 & 233 & 0.51 & -- \\ \hline
		\end{tabular}
	\end{center}
\caption{ \small{A comparison of verified observations and HPS RR Lyrae candidates. \label{tab:HPSlim15}}}
\end{table*}

%[46, 12, 19, 15, 22, 20, 22, 28, 28, 23, 9, 0]
%[92, 13, 19, 21, 14, 14, 19, 19, 11, 5, 6, 0]

Figure~\ref{fig:probrrHPS} shows no correlation between the HPS criteria and candidates verified to be RR Lyrae stars.  Of the 1.5 Billion stars identified in our FOV, we isolated the 320,000 most variable, 
%using the lower quartile variance, as described in~\cref{sec:cuts}.
as described in~\cref{sec:cuts}.  Determination of variability class for these objects in discussed in~\cref{sec:PA}.


%The further a star lands from a linear fit to this plot, the higher probability it is variable.  
%~[DESCRIBE LOWER QUARTILE METHOD]
%The star (group) histogram counts have a massive peak around 110 (g), 140 (r) and 125 (i), with low count peaks at N<50.
%Picking N>100 doesn't look like it misses many real stars.
%The plot of quartile variance in realstar.? (col 11-10 vs col 9) shows a 
%flattening out at g<13, and then increase by Poisson as 0.2*m Sort these in order of decreasing variability


A grouping and matching algorithm, written by J. Tonry~\cite{gri}, made it possible to isolate and group 
stars from various nights of observations.  
Implementation of logPr(rnd) allowed for complete confirmation of RR Lyrae candidates.  Only stars with different 
variable classifications, those having lower amplitude variations, would have been able to go
undetected.  Masking out regions of high aliasing reduced the need to run a more rigorous analysis, % of variable candidates,  
due to the statistically improbability of these sources being variable.  A total of 5,658 stars fell within 
the masked region, shown in Figure~\ref{RlogPr.png}.  FLS and FSS analysis identified 1,239 variable stars in our FOV.  
A defining characteristic of RR Lyrae is their variability periods, falling between 0.05 and 1.2 days.  Using this 
restriction, 244 confirmed to be RR Lyrae matched the HPS dataset.  Variability classification was confirmed by visually inspecting 
the light curves of all 244 RR Lyrae candidates as well as the remaining 995 unclassified variable stars.  Following this procedure gives us 
$100\%$ purity.

\begin{figure}[H]
 \centering
 	\includegraphics[width=3.35in]{figures/NEW/probrr_vs_HPS.png}
 \caption{\it \small{Evaluation of HPS RR Lyrae criteria, using matched RR Lyrae.}}
 \label{fig:probrrHPS}
\end{figure}

The same matching algorithm~\cite{gri} made comparing observations with other variable catalogs possible.  
To evaluate the HPS RR Lyrae criteria, our observations were matched to stars analyzed for variability, by HPS.  
Shown in Table~\ref{tab:HPSlim15} and Figure~\ref{fig:probrrHPS}, there is no correlation between HPS criteria and 
verified RR Lyrae stars.  HPS claims any star with an assigned $\rho_{RRLyrae} > 0.20$, is most likely an RR Lyrae.  Furthermore, 
if $\rho_{RRLyrae} > 0.05$, the candidate is definitively an RR Lyrae.  Figure~\ref{fig:probrrHPS} shows the fractional 
portion of observed RR Lyrae, that matched to stars that have been assigned $\rho_{RRLyrae}$, by HPS.
There is a distinct correlation between confirmed RR Lyrae and assigned $\rho_{RRLyrae}$ values.  However, over $16\%$ of 
all observed RR Lyrae, that matched to HPS stars, have $\rho_{RRLyrae} < 0.20$.  Similarly, $10\%$ of all matched RR Lyrae 
fall below $\rho_{RRLyrae} < 0.05$.  Of all 244 matched RR Lyrae, only 9 have the highest RR Lyrae 
probability assigned, $\rho_{RRLyrae} > 0.90$.


\begin{figure}[H]
 \centering
 	\includegraphics[width=3.35in]{figures/NEW/probnotrr_vs_HPS.png}
 \caption{\it \small{Evaluation of HPS RR Lyrae criteria, using non-RR Lyrae matched objects.}}
 \label{fig:probnotrrHPS}
\end{figure}



Shown in Figure~\ref{fig:probnotrrHPS}, there exists an inverse relationship, between non-RR Lyrae confirmed stars and HPS 
assigned $\rho_{RRLyrae}$.   
All stars confirmed as not begin RR Lyrae should have assigned $\rho_{RRLyrae} > 0.20$, according to HPS; yet $23\%$ of 
all matched stars do not meet this criteria.  Complete validity of HPS's $\rho_{RRLyrae}$ would be indicated by 0 verified, 
non-RR Lyrae stars, being matched with a $\rho_{RRLyrae} > 0.05$.  This is not the case, with $30\%$ of all matched stars begin 
confirmed non-RR Lyrae and having an assigned $\rho_{RRLyrae} > 0.05$.  While HPS's $\rho_{RRLyrae}$ may indicate if a star is 
an RR Lyrae, it is not sufficient for absolute determination.







\section{Discussion}


In order to evaluate the completeness of our results, comparisons needed to be made to other variable star catalogs.  Simbad~\cite{simbad} provided a list of variable stars within our FOV.  Pulsating sources encompasses all variable objects.  With 48 objects overlapping our FOV, shown in Figure~\ref{fig:simoverlap}, we achieved a completeness of $98\%$.

~[possible put two light curves in as examples of objects that overlap our field]\\

\begin{figure}[H]
 \centering
 %\caption{prob(f)}
 %\textbf{Simbad Completeness}\par\medskip
 	\includegraphics[width=3.35in]{figures/simbadoverlap.png}
 \caption{\it \small{Observation path shown in red, with Simbad pulsators in blue and RR Lyrae in green.}}
 \label{fig:simoverlap}
\end{figure}





\section*{Acknowledgments}
We would like to thank \input acknowledgement.tex  % input acknowledgement





\setlength{\parindent}{0cm}

\bibliography{biblio}


%\begin{thebibliography}{99}  % the trailing 99 controls some obscure format--just use	
%~\\% needed for spacing	
%\bibitem{PSimgpipe} Magnier E 2006 The Pan-STARRS PS1 image processing pipeline. In \textit{Proceedings of the Advanced Maui Optical and Space Surveillance Technologies Conf.} (ed. Ryan S), \textit{Wailea, Maui, HI, 10–14 September 2006,} p. E5. Kihei, HI: The Maui Economic Development Board.\\	
%\bibitem{simbad} Simbad Database	
%\bibitem{tonrypipe} Tonry JL, et al. 2012 The Pan-STARRS1 photometric system. \textit{Astrophys. J.} \textbf{750}, 99. doi:10.1088/0004-637X/750/2/99\\	
%\bibitem{gri} Tonry JL, er al. 2016 (personal communications January-April 2016)\\	
%\end{thebibliography}


\end{document}

